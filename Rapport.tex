\documentclass[a4paper,oneside,article]{memoir}

\usepackage[utf8]{inputenc}  % Korrekt håndtering af æ, ø og å
\usepackage[T1]{fontenc}  % Korrekt håndtering af æ, ø og å
\usepackage{microtype}  % Typografisk magi! Giver bl.a. pænere orddeling
\usepackage{graphicx}  % Gør det muligt at indsætte billeder
\usepackage{amsmath}  % Giver adgang til uundværlige matematikting
\usepackage{amssymb}
\usepackage{mathtools}  % Retter ting i ams og giver ekstra muligheder
\usepackage[danish]{babel}  % Danske betegnelser og orddeling
\renewcommand{\danishhyphenmins}{22}  % Bedre dansk orddeling
\usepackage{tikz}  % Gør det muligt at tegne grafikker
\usepackage{minitoc}  % Gør det muligt at lave små indholdsfortegnelser
\usepackage{hyperref}
\usepackage[margin=2.5cm]{geometry}
\usepackage{eufrak}
\usepackage{yfonts}
\usepackage{calligra}

\begin{document}
\author{Aksel W. Hallundbæk, Asger M. Bøgestrand, Oscar Fabricius}
\title{Experimental Physics II - Report 1 \\ 
\Large{The Fresnel Relations}}
\date{\today}
\maketitle

\chapter{Introduction}
\section{The law of reflection and Snell's law}
When light hits the interface between two media of different refractive indices $n_1$ and $n_2$, both refraction and reflection of light can occur. The law of reflection, equation (1), and Snell's law, equation (2) give the angles of reflection and refraction for given angle of incidence and refractive indices. 

\begin{equation} \label{eq1}
\begin{split}
\theta_{incidence} = \theta_{reflection}
\end{split}
\end{equation}
\begin{equation} \label{eq2}
\begin{split}
n_1\text{sin}\theta_1 = n_2\text{sin}\theta_2
\end{split}
\end{equation}

\begin{figure}[h]
\centering
\includegraphics[width=0.4\textwidth]{snellslaw.png}
\caption{Reflection of light at the interface between two media.}
\end{figure}

\section{Polarization}
Light consists of mutually perpendicular oscillating electric and magnetic fields. The orientation of these oscillations is called the polarization of light. Unpolarized light consists of fields oscillating in all possible orientations, however any polarization can be broken into a combination of two orthogonal linear polarizations, i.e. if we can determine the behavior of light of each linear polarization, we can describe the behavior of light of any (complicated) polarization. 

Typically these orthogonal linear polarizations are chosen to be so-called s- and p-polarized light. The plane spanned by the incoming beam of light and the normal vector of the surface at the point of incidence is called the plane of incidence. If the polarization of the light is parallel to the plane of incidence, it is p-polarized. If the polarization of the light is perpendicular to the plane of incidence, it is s-polarized. 

TILFØJ EN FIGUR MÅSKE

\section{The Fresnel relations}
We are interested in the intensities of transmitted (refracted) and reflected light for s-polarized and p-polarized light respectively. The Fresnel relations give the amplitudes of transmitted and reflected light for the two polarizations, and from these one can find the corresponding intensities:

\begin{equation} \label{eq3}
\begin{split}
R_p = \frac{\text{tan}^2(\theta_1-\theta_2)}{\text{tan}^2(\theta_1+\theta_2)}
\end{split}
\end{equation}
\begin{equation} \label{eq4}
\begin{split}
T_p = \frac{\text{sin}2\theta_1\text{sin}2\theta_2}{\text{sin}^2(\theta_1+\theta_2)\text{cos}^2(\theta_1-\theta_2)}
\end{split}
\end{equation}
\begin{equation} \label{eq5}
\begin{split}
R_s = \frac{\text{sin}^2(\theta_1-\theta_2)}{\text{sin}^2(\theta_1+\theta_2)}
\end{split}
\end{equation}
\begin{equation} \label{eq6}
\begin{split}
T_p = \frac{\text{sin}2\theta_1\text{sin}2\theta_2}{\text{sin}^2(\theta_1+\theta_2)}
\end{split}
\end{equation}

Here $R$ and $T$ are the intensities of reflected and transmitted light and the indices $s$ and $p$ stand for s-polarized and p-polarized light. 

\section{Brewster's angle and the critical angle}
For p-polarized light, there exists the so-called Brewster's angle, at which there is no reflected light. Brewster's angle is given by $R_p(\theta_B)=0$. There is no reflected light, when $\theta_1+\theta_2=\frac{\pi}{2}$. From Snell's law we then have:

\begin{equation}
\begin{split}
n_1\text{sin}\theta_B=n_2\text{sin}(\frac{\pi}{2}-\theta_B)=n_2\text{cos}\theta_B \\
\text{tan}\theta_B = \frac{n_2}{n_1}\\
\theta_B = \text{arctan}(\frac{n_2}{n_1}) 
\end{split}
\end{equation}

When light transitions from a medium of higher refractive index to one of lower refractive index there exists the critical angle for total internal reflection, i.e. no transmitted light. This occurs for both s- and p-polarized light. This angle, $\theta_C$ follows directly from Snell's law with $\theta_2 = \frac{\pi}{2}$:

\begin{equation}
\begin{split}
\theta_C = \text{arcsin}(\frac{n_2}{n_1})
\end{split}    
\end{equation}

\section{Experiment FIND BEDRE TITEL}
In case one knows the refractive index of one medium and can measure the four intensities of reflected and transmitted light for the two polarizations respectively as a function of the angle of incidence, one can determine the refractive index, Brewster's angle and the critical angle for the given second medium. This will be further discussed in the following section. 

INDSÆT MÅSKE FIGURER DER VISER GRAFER FOR INTENSITETER OG DE SPÆNDENDE VINKLER.

\chapter{Experimental setup}

\section{Measurment Plan}
\begin{itemize}
\item First lab session \\
For the first lab session, we measured the refractive index of the glass prism.
Then we measured the reflected intensity from air to glass. After the session we realized that the background intensity was changing with each measurement,
and therefore we decided to improve our method for the second lab session.
\item Second lab session \\
For the second lab session, we improved our experimental method, and measured the reflected intensisty of both s and p polarized light, from air to glass and glass to air.
\item Third lab session \\
For the third lab session, we measured the transmitted intensity of both s and p polarized light, both form air to glass and from glass to air.
\end{itemize}

\section{Equipment}
\begin{itemize}
\item Laser
\item Collimating Slits
\item Two Polarizers
\item Glass prism
\item Focusing Lens
\item Photodetector
\end{itemize}

\section{General set-up and safety}
The general set-up for the experiment is shown in figure 1. 
The laser is placed on a rail, and the beam is collimated using the collimating slits. 
The beam then passes through one polarizer at 45 degrees to ensure equal parts of s and p polarized light. 
The beam then hits the glass prism, where it is either reflected or transmitted. The reflected or transmitted beam passes through a second polarizer to ensure that only one polarization component is measured, and is then focused onto a photodetector using a focusing lens.
Both the prism and the photodetector are mounted on rotatable mounts, so that the angle of incidence can be varied.
The laser is placed in parrallel to the table, and the beam is therefore not directed towards the eyes of the experimenters. The laser is turned on only when measurements are being taken, and care is taken to ensure that the beam does not reflect off any surfaces in a way that could direct it towards the eyes of the experimenters.


\begin{figure}[h]
\centering
\includegraphics[width=0.8\textwidth]{setup.JPG}
\caption{General set-up for the experiment.}
\end{figure}


\section{Method}
The intensities of s- and p-polarized striking the detector for a series of different incident angles was measured. 
For each measurement the background intensity was noted, 
since the background intensity could depend on the orientation of the detector with respect to the windows and lamps. 
Intensities were measured first for light going from air to glass and later vice versa. 
This can be done since the prism is semicircular. Only on the flat surface will light be reflected and refracted. 
On the circular boundary, the angle of incidence, reflection and refraction will always be zero,
 i.e. the interesting transition from one medium to another can be chosen simply by turning the prism 180 degrees. 
 
 After setting the angle of incidence, the detector was moved to find the position of peak intensity for the reflected or transmitted beam.
 This was done by slowly moving the detector through the refelcted or refracted beam, noting the peak intensity,
 and then moving the detector back to that position of peak intensity. 


\chapter{Results}

\chapter{Discussion}
Our results generally follow the expected trends for both s- and p-polarized and for both air to glas and glass to air.
- Vinklerne er lidt off på nogle, hvilket kan skyldes at prismen ikke var centreret, laserens udstrækning. 
- Transmission ligger en del under, da prismet er uperfekt. Der tabes intensitet.



\end{document}